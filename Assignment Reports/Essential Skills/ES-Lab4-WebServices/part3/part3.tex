\newpage
\section{Web-services}
\begin{itemize}
\item are a service offered by an electronic device to another electronic device, communicating with each other via the World Wide Web
\item are \textcolor{b}{platform-independent} and \textcolor{b}{language-independent}
\item use \textcolor{b}{HTTP} for transmitting messages (such as the service request and response)
\end{itemize}

\subsection*{SOAP Based Web services}
\text{Enabling Technologies}
\begin{itemize}
\item[--] \textcolor{g}{XML}: \textbf{Extensible Markup Language} is a markup language that defines a set of rules for encoding documents in a format that is both human-readable and machine-readable. 
\item[--] \textcolor{g}{SOAP}: \textbf{Simple Object Access Protocol} is an XML-based lightweight protocol for the exchange of information in a decentralized, distributed environment.
\item[--] \textcolor{g}{WSDL}: \textbf{Web Services Description Language} is an XML vocabulary that provides a standard way of describing service interface definition languages (IDLs).
\end{itemize}
\begin{Figure}
\centering
\includegraphics[width=5.5cm]{ws}
\captionof{figure}{Enabling Technologies}
\label{fig:SOAP}
\end{Figure}

\subsection*{Web Service Reference Framework}

\begin{center}
\textbf{WSRF specification}
{\scriptsize
\begin{Table}
\begin{tabular}{|K{.45\textwidth}|K{.45\textwidth}|}
\hline
\textbf{Name} & \textbf{Describes}\\[.4em]
\hline
WS-ResourceProperties &  specifies how resource properties are defined and accessed\\[.4em]
\hline
WS-ResourceLifetime & supplies some basic mechanisms to manage the lifecycle of our resources\\[.4em]
\hline
WS-ServiceGroup & specifies how exactly we should go about grouping services or WS-Resources together\\[.4em]
\hline
WS-BaseFaults & aims to provide a standard way of reporting faults when something goes wrong during a WS-Service invocation\\[.4em]
\hline
\end{tabular}
\captionof{table}{The five specification composing the WSRF}
\label{tbl:WSPF}
\end{Table}
}
\end{center}

\subsection*{RESTful Web service}
\begin{itemize}
\item stands for \textcolor{g}{Re}presentaLonal \textcolor{g}{S}tate \textcolor{g}{T}ransfer
\item was coined by	Roy	Fielding in	2000 in his Ph.D. dissertaLon	to	describe a	design pattern for implementing networked systems
\item in many ways,	the	World Wide Web itself, based on HTTP, can be viewed as a REST-based architecture
\end{itemize}

\begin{Figure}
\centering
\includegraphics[width=\linewidth]{REST.png}
\captionof{figure}{The REST way of Designing the Web Services}
\label{fig:SOAP}
\end{Figure}
