\newpage
\section{Version Control System}
\begin{multicols*}{2}
A \textbf{version control system} provides \textcolor{red}{support} for the \textcolor{red}{development} of \textcolor{red}{software} in particular.
\begin{itemize}
\item documenting an annotated history of a project is a \textbf{\textit{complex matter}} (changes, bugs)
\item multiple editors in use, keep developers from overwriting each other work
\item \textbf{\textit{backtracking}} to a previous version
\item \textbf{\textit{branching}}
\end{itemize}
\begin{center}
\textcolor{b}{\textbf{VCS}}
\begin{Table}
\begin{tabular}{K{.45\textwidth}|K{.45\textwidth}}
\hline
\textbf{YES} & \textbf{NO}\\[.4em]
\hline
facilitates \textcolor{g}{bug detection} when software is modified & \vspace{0.2em} a \textcolor{g}{build} system\\[.4em]
\hline
\vspace{0.6em}\textcolor{g}{economy} in disk space while saving versions & a \textcolor{g}{substitute} for communication between developers of for management\\[.4em]
\hline
\textcolor{g}{prevents code over-writing} in a project team & \textcolor{g}{create any magic} for you\\[.4em]
\hline
\end{tabular}
\captionof{table}{VCS do’s and don'ts ...}
\label{tbl_ex}
\end{Table}
\end{center}

\textbf{Version Control System Types [\ref{fig:vcs}]:}
\begin{enumerate}
\item Centralized (RCS, CVS, SVN)
\item Decentralized (Git, Mercurial, Bazaar)
\end{enumerate}

\textcolor{b}{Centralized VCS}

In a centralized VCS, there is a \textbf{single copy} of the repository, on a server. The developers check out the source code from the central repository. 

\textit{Disadvantages:}
\begin{itemize}
\item [--] If the server goes down, \textbf{nobody can version the changes} they worked at!
\item [--] If the server's hard disk becomes corrupt, all the \textbf{history on the changes is lost}!
\end{itemize}

\textcolor{b}{Distributed VCS}

In a distributed VCS, every developer \textbf{clones the repository}, not only it gets the project files.
Another feature is that the change sets can be \textbf{committed locally}, and then \textbf{push the changes} to the remote servers.

\textit{Advantages:}
\begin{itemize}
\item [--] In case the server dies, any of the client repositories can be used to restore it.
\item [--] No network connection needed for operations other than push and pull.
\end{itemize}
\textit{Disadvantages:}
\begin{itemize}
\item [--] If the project contains many large files, the space needed to store all the versions for those files increases
\item [---] If the project has a very long history, downloading the entire history can take more time
\end{itemize}

\begin{Figure}
\includegraphics[width=\textwidth]{image}
\captionof{figure}{VCS. Centralized \textit{vs} Distributed}
\label{fig:vcs}
\end{Figure}

\end{multicols*}